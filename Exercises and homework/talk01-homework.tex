% Options for packages loaded elsewhere
\PassOptionsToPackage{unicode}{hyperref}
\PassOptionsToPackage{hyphens}{url}
%
\documentclass[
]{ctexart}
\usepackage{amsmath,amssymb}
\usepackage{lmodern}
\usepackage{ifxetex,ifluatex}
\ifnum 0\ifxetex 1\fi\ifluatex 1\fi=0 % if pdftex
  \usepackage[T1]{fontenc}
  \usepackage[utf8]{inputenc}
  \usepackage{textcomp} % provide euro and other symbols
\else % if luatex or xetex
  \usepackage{unicode-math}
  \defaultfontfeatures{Scale=MatchLowercase}
  \defaultfontfeatures[\rmfamily]{Ligatures=TeX,Scale=1}
\fi
% Use upquote if available, for straight quotes in verbatim environments
\IfFileExists{upquote.sty}{\usepackage{upquote}}{}
\IfFileExists{microtype.sty}{% use microtype if available
  \usepackage[]{microtype}
  \UseMicrotypeSet[protrusion]{basicmath} % disable protrusion for tt fonts
}{}
\makeatletter
\@ifundefined{KOMAClassName}{% if non-KOMA class
  \IfFileExists{parskip.sty}{%
    \usepackage{parskip}
  }{% else
    \setlength{\parindent}{0pt}
    \setlength{\parskip}{6pt plus 2pt minus 1pt}}
}{% if KOMA class
  \KOMAoptions{parskip=half}}
\makeatother
\usepackage{xcolor}
\IfFileExists{xurl.sty}{\usepackage{xurl}}{} % add URL line breaks if available
\IfFileExists{bookmark.sty}{\usepackage{bookmark}}{\usepackage{hyperref}}
\hypersetup{
  pdftitle={talk01 练习与作业},
  hidelinks,
  pdfcreator={LaTeX via pandoc}}
\urlstyle{same} % disable monospaced font for URLs
\usepackage{color}
\usepackage{fancyvrb}
\newcommand{\VerbBar}{|}
\newcommand{\VERB}{\Verb[commandchars=\\\{\}]}
\DefineVerbatimEnvironment{Highlighting}{Verbatim}{commandchars=\\\{\}}
% Add ',fontsize=\small' for more characters per line
\usepackage{framed}
\definecolor{shadecolor}{RGB}{248,248,248}
\newenvironment{Shaded}{\begin{snugshade}}{\end{snugshade}}
\newcommand{\AlertTok}[1]{\textcolor[rgb]{0.94,0.16,0.16}{#1}}
\newcommand{\AnnotationTok}[1]{\textcolor[rgb]{0.56,0.35,0.01}{\textbf{\textit{#1}}}}
\newcommand{\AttributeTok}[1]{\textcolor[rgb]{0.77,0.63,0.00}{#1}}
\newcommand{\BaseNTok}[1]{\textcolor[rgb]{0.00,0.00,0.81}{#1}}
\newcommand{\BuiltInTok}[1]{#1}
\newcommand{\CharTok}[1]{\textcolor[rgb]{0.31,0.60,0.02}{#1}}
\newcommand{\CommentTok}[1]{\textcolor[rgb]{0.56,0.35,0.01}{\textit{#1}}}
\newcommand{\CommentVarTok}[1]{\textcolor[rgb]{0.56,0.35,0.01}{\textbf{\textit{#1}}}}
\newcommand{\ConstantTok}[1]{\textcolor[rgb]{0.00,0.00,0.00}{#1}}
\newcommand{\ControlFlowTok}[1]{\textcolor[rgb]{0.13,0.29,0.53}{\textbf{#1}}}
\newcommand{\DataTypeTok}[1]{\textcolor[rgb]{0.13,0.29,0.53}{#1}}
\newcommand{\DecValTok}[1]{\textcolor[rgb]{0.00,0.00,0.81}{#1}}
\newcommand{\DocumentationTok}[1]{\textcolor[rgb]{0.56,0.35,0.01}{\textbf{\textit{#1}}}}
\newcommand{\ErrorTok}[1]{\textcolor[rgb]{0.64,0.00,0.00}{\textbf{#1}}}
\newcommand{\ExtensionTok}[1]{#1}
\newcommand{\FloatTok}[1]{\textcolor[rgb]{0.00,0.00,0.81}{#1}}
\newcommand{\FunctionTok}[1]{\textcolor[rgb]{0.00,0.00,0.00}{#1}}
\newcommand{\ImportTok}[1]{#1}
\newcommand{\InformationTok}[1]{\textcolor[rgb]{0.56,0.35,0.01}{\textbf{\textit{#1}}}}
\newcommand{\KeywordTok}[1]{\textcolor[rgb]{0.13,0.29,0.53}{\textbf{#1}}}
\newcommand{\NormalTok}[1]{#1}
\newcommand{\OperatorTok}[1]{\textcolor[rgb]{0.81,0.36,0.00}{\textbf{#1}}}
\newcommand{\OtherTok}[1]{\textcolor[rgb]{0.56,0.35,0.01}{#1}}
\newcommand{\PreprocessorTok}[1]{\textcolor[rgb]{0.56,0.35,0.01}{\textit{#1}}}
\newcommand{\RegionMarkerTok}[1]{#1}
\newcommand{\SpecialCharTok}[1]{\textcolor[rgb]{0.00,0.00,0.00}{#1}}
\newcommand{\SpecialStringTok}[1]{\textcolor[rgb]{0.31,0.60,0.02}{#1}}
\newcommand{\StringTok}[1]{\textcolor[rgb]{0.31,0.60,0.02}{#1}}
\newcommand{\VariableTok}[1]{\textcolor[rgb]{0.00,0.00,0.00}{#1}}
\newcommand{\VerbatimStringTok}[1]{\textcolor[rgb]{0.31,0.60,0.02}{#1}}
\newcommand{\WarningTok}[1]{\textcolor[rgb]{0.56,0.35,0.01}{\textbf{\textit{#1}}}}
\usepackage{graphicx}
\makeatletter
\def\maxwidth{\ifdim\Gin@nat@width>\linewidth\linewidth\else\Gin@nat@width\fi}
\def\maxheight{\ifdim\Gin@nat@height>\textheight\textheight\else\Gin@nat@height\fi}
\makeatother
% Scale images if necessary, so that they will not overflow the page
% margins by default, and it is still possible to overwrite the defaults
% using explicit options in \includegraphics[width, height, ...]{}
\setkeys{Gin}{width=\maxwidth,height=\maxheight,keepaspectratio}
% Set default figure placement to htbp
\makeatletter
\def\fps@figure{htbp}
\makeatother
\setlength{\emergencystretch}{3em} % prevent overfull lines
\providecommand{\tightlist}{%
  \setlength{\itemsep}{0pt}\setlength{\parskip}{0pt}}
\setcounter{secnumdepth}{5}
\ifluatex
  \usepackage{selnolig}  % disable illegal ligatures
\fi

\title{talk01 练习与作业}
\author{}
\date{\vspace{-2.5em}}

\begin{document}
\maketitle

{
\setcounter{tocdepth}{2}
\tableofcontents
}
\hypertarget{ux7ec3ux4e60ux548cux4f5cux4e1aux8bf4ux660e}{%
\subsection{练习和作业说明}\label{ux7ec3ux4e60ux548cux4f5cux4e1aux8bf4ux660e}}

将相关代码填写入以 ```\{r\} ``` 标志的代码框中,运行并看到正确的结果;

完成后,用工具栏里的``Knit''按键生成PDF文档;

\textbf{将生成的PDF}改为:\textbf{\texttt{姓名-学号-talk01作业.pdf}},并提交到老师指定的平台/钉群。

\hypertarget{talk01-ux5185ux5bb9ux56deux987e}{%
\subsection{talk01 内容回顾}\label{talk01-ux5185ux5bb9ux56deux987e}}

\begin{itemize}
\item
  R背景介绍
\item
  几个简单示例

  \begin{itemize}
  \tightlist
  \item
    R流行度调查
  \item
    R job trends
  \end{itemize}
\end{itemize}

\hypertarget{ux7ec3ux4e60ux548cux4f5cux4e1aux8bf4ux660e-1}{%
\subsection{练习和作业说明}\label{ux7ec3ux4e60ux548cux4f5cux4e1aux8bf4ux660e-1}}

将相关代码填写入以 ```\{r\} ``` 标志的代码框中,运行并看到正确的结果;

完成后,用工具栏里的\textbf{Knit}按键生成PDF文档;

将得到的PDF文件\textbf{更名}为:\texttt{姓名-学号-talk\#\#作业.pdf}。

将word文档提交到老师指定的平台/钉群。

\hypertarget{ux7ec3ux4e60ux4f5cux4e1a1}{%
\subsection{练习/作业1}\label{ux7ec3ux4e60ux4f5cux4e1a1}}

安装 \texttt{R}和\texttt{RStudio}最新版;

用\texttt{RStudio}打开本脚本,按提示操作安排相应的包;

成功运行本脚本,并出输出作业;

\hypertarget{ux7ec3ux4e60ux4f5cux4e1a2}{%
\subsection{练习/作业2}\label{ux7ec3ux4e60ux4f5cux4e1a2}}

重现talk01中的两个图。

\begin{enumerate}
\def\labelenumi{\arabic{enumi}.}
\item
  安装需要的包:
\item
  R 的流行性调查
\end{enumerate}

运行以下代码,得到R的流行性调查;

注意输入文件:\texttt{chaper01\_preface\_scholarly\_impact\_2012.4.9.csv}
的正确路径;

注意理解每行代码的意义;

\begin{Shaded}
\begin{Highlighting}[]
\FunctionTok{library}\NormalTok{(}\StringTok{"ggplot2"}\NormalTok{); }\FunctionTok{library}\NormalTok{(}\StringTok{"reshape2"}\NormalTok{);}

\NormalTok{dat }\OtherTok{\textless{}{-}} \FunctionTok{read.csv}\NormalTok{(}\AttributeTok{file =} \StringTok{"../data/talk01/chaper01\_preface\_scholarly\_impact\_2012.4.9.csv"}\NormalTok{);}

\NormalTok{cols.subset }\OtherTok{\textless{}{-}} \FunctionTok{c}\NormalTok{(}\StringTok{"Year"}\NormalTok{, }\StringTok{"JMP"}\NormalTok{,}\StringTok{"Minitab"}\NormalTok{,}\StringTok{"Stata"}\NormalTok{,}\StringTok{"Statistica"}\NormalTok{,}\StringTok{"Systat"}\NormalTok{,}\StringTok{"R"}\NormalTok{);}
\NormalTok{Subset }\OtherTok{\textless{}{-}}\NormalTok{ dat[ , cols.subset];}
\NormalTok{ScholarLong }\OtherTok{\textless{}{-}} \FunctionTok{melt}\NormalTok{(Subset, }\AttributeTok{id.vars =} \StringTok{"Year"}\NormalTok{);}
\FunctionTok{names}\NormalTok{(ScholarLong) }\OtherTok{\textless{}{-}} \FunctionTok{c}\NormalTok{(}\StringTok{"Year"}\NormalTok{,}\StringTok{"Software"}\NormalTok{, }\StringTok{"Hits"}\NormalTok{);}

\NormalTok{plot1 }\OtherTok{\textless{}{-}} 
  \FunctionTok{ggplot}\NormalTok{(ScholarLong, }\FunctionTok{aes}\NormalTok{(Year, Hits, }\AttributeTok{group=}\NormalTok{Software)) }\SpecialCharTok{+} \CommentTok{\#准备}
    \FunctionTok{geom\_smooth}\NormalTok{(}\FunctionTok{aes}\NormalTok{(}\AttributeTok{fill=}\NormalTok{Software), }\AttributeTok{position=}\StringTok{"fill"}\NormalTok{, }\AttributeTok{method=}\StringTok{"loess"}\NormalTok{) }\SpecialCharTok{+} \CommentTok{\#画图}
    \FunctionTok{ggtitle}\NormalTok{(}\StringTok{"Market share"}\NormalTok{) }\SpecialCharTok{+} \CommentTok{\#设置图标题}
    \FunctionTok{scale\_x\_continuous}\NormalTok{(}\StringTok{"Year"}\NormalTok{) }\SpecialCharTok{+} \CommentTok{\# 改变X轴标题}
    \FunctionTok{scale\_y\_continuous}\NormalTok{(}\StringTok{"Google Scholar \%"}\NormalTok{, }\AttributeTok{labels =} \ConstantTok{NULL}\NormalTok{ ) }\SpecialCharTok{+}
    \FunctionTok{theme}\NormalTok{(}\AttributeTok{axis.ticks =} \FunctionTok{element\_blank}\NormalTok{(),  }\AttributeTok{text =} \FunctionTok{element\_text}\NormalTok{(}\AttributeTok{size=}\DecValTok{14}\NormalTok{)) }\SpecialCharTok{+} 
    \FunctionTok{guides}\NormalTok{(}\AttributeTok{fill=}\FunctionTok{guide\_legend}\NormalTok{( }\AttributeTok{title =} \StringTok{"Software"}\NormalTok{,  }\AttributeTok{reverse =}\NormalTok{ F )) }\SpecialCharTok{+} 
    \FunctionTok{geom\_text}\NormalTok{(}\AttributeTok{data =} \FunctionTok{data.frame}\NormalTok{( }\AttributeTok{Year =} \DecValTok{2011}\NormalTok{,  }\AttributeTok{Software =} \StringTok{"R"}\NormalTok{, }\AttributeTok{Hits =} \FloatTok{0.10}\NormalTok{ ),}
              \FunctionTok{aes}\NormalTok{(}\AttributeTok{label =}\NormalTok{ Software), }\AttributeTok{hjust =} \DecValTok{0}\NormalTok{, }\AttributeTok{vjust =} \FloatTok{0.5}\NormalTok{);}

\NormalTok{plot1; }\DocumentationTok{\#\# 画图}
\end{Highlighting}
\end{Shaded}

\begin{verbatim}
## `geom_smooth()` using formula 'y ~ x'
\end{verbatim}

\begin{verbatim}
## Warning: Stacking not well defined when not anchored on the axis
\end{verbatim}

\includegraphics{talk01-homework_files/figure-latex/unnamed-chunk-2-1.pdf}

\begin{enumerate}
\def\labelenumi{\arabic{enumi}.}
\setcounter{enumi}{2}
\tightlist
\item
  R 的招聘趋势
\end{enumerate}

运行以下代码,得到R 的招聘趋势

注意输入文件:\texttt{chaper01\_preface\_scholarly\_impact\_2012.4.9.csv}
的正确路径;

注意理解每行代码的意义;

\begin{Shaded}
\begin{Highlighting}[]
\FunctionTok{library}\NormalTok{(}\StringTok{"ggplot2"}\NormalTok{); }\DocumentationTok{\#\# 主作图包}

\DocumentationTok{\#\#2. {-}{-} 读取数据 {-}{-}}
\NormalTok{dat }\OtherTok{\textless{}{-}} \FunctionTok{read.table}\NormalTok{(}\AttributeTok{file =}\StringTok{"../data/talk01/chaper01\_preface\_indeed\_com\_stats\_2015.txt"}\NormalTok{, }
                  \AttributeTok{header =}\NormalTok{ T, }\AttributeTok{as.is =}\NormalTok{ T);}
\DocumentationTok{\#\#3. 处理数据}
\NormalTok{dat}\SpecialCharTok{$}\NormalTok{date }\OtherTok{\textless{}{-}} \FunctionTok{as.Date}\NormalTok{(dat}\SpecialCharTok{$}\NormalTok{date); }\DocumentationTok{\#\# 把第一列改为日期}

\CommentTok{\#根据job对software进行调整}
\NormalTok{dat }\OtherTok{\textless{}{-}} \FunctionTok{transform}\NormalTok{(dat, }\AttributeTok{software =} \FunctionTok{reorder}\NormalTok{(software, job)); }

\NormalTok{plot2 }\OtherTok{\textless{}{-}}
  \FunctionTok{ggplot}\NormalTok{( dat, }\FunctionTok{aes}\NormalTok{( date, job, }\AttributeTok{group =}\NormalTok{ software, }\AttributeTok{colour =}\NormalTok{ software) ) }\SpecialCharTok{+}
    \FunctionTok{geom\_line}\NormalTok{( }\AttributeTok{size =} \FloatTok{0.8}\NormalTok{  ) }\SpecialCharTok{+}
    \FunctionTok{ggtitle}\NormalTok{(}\StringTok{"Job trends (data from indeed.com)"}\NormalTok{) }\SpecialCharTok{+} \CommentTok{\#设置图标题}
    \FunctionTok{xlab}\NormalTok{(}\StringTok{"Year"}\NormalTok{) }\SpecialCharTok{+} \FunctionTok{ylab}\NormalTok{(}\StringTok{"\%"}\NormalTok{) }\SpecialCharTok{+}
    \CommentTok{\#改变字体大小;要放在theme\_grey()后面  }
    \FunctionTok{theme}\NormalTok{( }\AttributeTok{text =} \FunctionTok{element\_text}\NormalTok{(}\AttributeTok{size=}\DecValTok{14}\NormalTok{) ) }\SpecialCharTok{+} 
    \FunctionTok{guides}\NormalTok{(}\AttributeTok{colour=}\FunctionTok{guide\_legend}\NormalTok{( }\AttributeTok{title =} \StringTok{"Tool"}\NormalTok{,  }\AttributeTok{reverse =} \ConstantTok{TRUE}\NormalTok{ )) }\SpecialCharTok{+}
    \FunctionTok{scale\_colour\_brewer}\NormalTok{(}\AttributeTok{palette=}\StringTok{"Set1"}\NormalTok{) }\SpecialCharTok{+} \CommentTok{\#改变默认颜色}
    \FunctionTok{geom\_text}\NormalTok{(}\AttributeTok{data =}\NormalTok{ dat[dat}\SpecialCharTok{$}\NormalTok{date }\SpecialCharTok{==} \StringTok{"2015{-}01{-}01"} \SpecialCharTok{\&}\NormalTok{ dat}\SpecialCharTok{$}\NormalTok{software }\SpecialCharTok{\%in\%} \FunctionTok{c}\NormalTok{(}\StringTok{"R"}\NormalTok{), ], }
              \FunctionTok{aes}\NormalTok{(}\AttributeTok{label =}\NormalTok{ software), }\AttributeTok{hjust =} \DecValTok{0}\NormalTok{, }\AttributeTok{vjust =} \FloatTok{0.5}\NormalTok{);}

\NormalTok{plot2;}
\end{Highlighting}
\end{Shaded}

\includegraphics{talk01-homework_files/figure-latex/unnamed-chunk-3-1.pdf}

\end{document}
